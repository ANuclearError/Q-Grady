\documentclass[report.tex]{subfiles}
\begin{document}

\chapter{Related Work} % (fold)
\label{cha:related_work}

% ==============================================================================
% ==============================================================================

\section{Non-Local Boxes} % (fold)
\label{sec:non_local_boxes}
The study of non-local boxes was introduced by Sandu Popescu and Daniel
Rohrlich, physicists in 1993 and 1994, as part of their research on quantum
non-locality was inspired by the CHSH inequality by Causer et al.

Consider a device with two ends, held by Alice and Bob. Each provides an
input (\(x\) and \(y)\), and observe the output produced by the box (\(a\) and
\(b)\). Based on some probability distribution \(P(a,b | x,y)\) where \(a, b, x,
y \in {0, 1}\). This device will immediately produce an output when given an
input, preventing any communication as a result of a delayed input. These
devices are non-signalling, forbidding faster than light communication and
ensuring that Alice or Bob cannot learn anything from the other party's input.

\subsection{Non-Signalling} % (fold)
\label{sub:non_signalling}

% subsection non_signalling (end)
% section non_local_boxes (end)

% ==============================================================================

\section{Prism} % (fold)
\label{sec:prism}
% section prism (end)
% chapter related_work (end)

\newpage
\end{document}