\documentclass[report.tex]{subfiles}
\begin{document}

\chapter{Related Work} % (fold)
\label{cha:related_work}

% ==============================================================================
% ==============================================================================

\section{Non-Local Boxes} % (fold)
\label{sec:non_local_boxes}
Consider a device with two ends, one of which is held by Alice, and the other of
which is held by Bob. Each end has an input that can be provided by Alice (which
we will call \(x\)) or Bob (which we will call \(y\), and an output that each
can observe (\(a\) and \(b\) respectively. This would be our box, and it will
output \(a\) and \(b\) based on some probability distribution \(P(a, b | x, y)\)
where \(a, b, x, y \in {0, 1}\).

An example of such a box would be the Popescu/Rohrlich (PR) box, the probability
distribution of which is provided below, where the row specifies the input \(x\)
and \(y\), and the columns specifies the output \(a\) and \(b\).

\begin{center}
\begin{tabular}{l | c c c c}
  & 00 & 01 & 10 & 11 \\
  \hline
  00 & \(\frac{1}{2}\) & 0 & 0 & \(\frac{1}{2}\) \\
  01 & \(\frac{1}{2}\) & 0 & 0 & \(\frac{1}{2}\) \\
  10 & \(\frac{1}{2}\) & 0 & 0 & \(\frac{1}{2}\) \\
  11 & 0 & \(\frac{1}{2}\) & \(\frac{1}{2}\) & 0 \\
\end{tabular}
\end{center}

The PR box can be described as a (2, 2, 2) set-up, meaning that it has two ends,
two inputs and two outputs. It is possible for there to be non-local boxes with
different set-ups, although these are more complex to work with. Note that the
probability of the PR box \(P(a, b | x, y\) can be simplified as:
\[
    P(a, b | x, y) = 
    \begin{cases}
        \frac{1}{2} & \quad a \text{ XOR } B = x \text{ AND } y \\
        0 & \quad \text{otherwise} \\
    \end{cases}
\]

\subsection{Non-Signalling} % (fold)
\label{sub:non_signalling}
As essential part of the definition of a non-local box is the non-signalling
property. A non-signalling box means that Alice could not use the value of \(a\)
to determine the value of \(y\) in the above configuration. This condition
allows for the probability distribution \(P(a, b | x, y)\) to be reduced to
simply \(P(a|x)\).

To ensure non-signalling in a (2, 2, 2) set- up, the following conditions must
be met:

\[\sum_{b} P(a, b | x, y) = \sum_{b} P(a, b | x, y') = P(a | x) 
\quad \forall a, x, y, y'\]
\[\sum_{a} P(a, b | x, y) = \sum_{a} P(a, b | x', y) = P(b | y) 
\quad \forall b, y, x, x'\]

The PR-box is one such example of a non-signalling box.

% ==============================================================================

% subsection non_signalling (end)
% section non_local_boxes (end)
\section{Prism} % (fold)
\label{sec:prism}
PRISM is a probabilistic model checker, software used to create formal models
using the PRISM language and allows for analysis of the models created. The
models are created to represent systems with random or probabilistic behaviour.

Non-local boxes can be created using the PRISM language, although it is not the
most elegant system.
% section prism (end)
% chapter related_work (end)

\newpage
\end{document}