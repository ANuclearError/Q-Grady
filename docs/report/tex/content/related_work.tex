\documentclass[report.tex]{subfiles}
\begin{document}

\chapter{Related Work} % (fold)
\label{cha:related_work}

% ==============================================================================
% ==============================================================================

\section{Non-Local Boxes} % (fold)
\label{sec:non_local_boxes}
The study of non-local boxes was introduced by Sandu Popescu and Daniel
Rohrlich, physicists in 1993 and 1994, as part of their research on quantum
non-locality was inspired by the CHSH inequality by Causer et al.

Consider a device with two ends, held by Alice and Bob. Each provides an
input (\(x\) and \(y)\), and observe the output produced by the box (\(a\) and
\(b)\). Based on some probability distribution \(P(a,b | x,y)\) where \(a, b, x,
y \in {0, 1}\) (\cite[Definition~1]{nlb_lamontagne}). 

This device will immediately produce an output when given an input, preventing
any communication as a result of a delayed input. These devices are
non-signalling, forbidding faster than light communication and ensuring that
Alice or Bob cannot learn anything from the other party's input.

\subsection{Non-Signalling} % (fold)
\label{sub:non_signalling}
As mentioned in Section \ref{sec:non_local_boxes}, an important part of the
non-local box is that it is non-signalling. Non-signalling means that in the
described experiment, Alice can not be made aware of Bob's choice of input or
vice-versa. This requirement imposes restrictions on the probability
distribution, since the probability of an output should depend solely on its
respective input. These restrictions allow for a represented of a reduced
probability on the outputs as well.

The non-signalling restriction can be represented by the given formulae
(\cite[Section~II.A]{PhysRevA.71.022101}):
\[\sum_{b} P(a, b | x, y) = \sum_{b} P(a, b | x, y') = P(a | x) 
\quad \forall a, x, y, y'\]
\[\sum_{a} P(a, b | x, y) = \sum_{a} P(a, b | x', y) = P(b | y) 
\quad \forall b, y, x, x'\]
% subsection non_signalling (end)


\subsection{PR-Box} % (fold)
\label{sub:pr_box}
The PR-box, originally proposed by and named after Popescu-Rorlich, can be
considered the `Hello World' of non-local boxes. It is the most basic structure,
with two inputs and two outputs, and is the basis for this area of research.

The PR-Box has the following probability distribution:
\[
    P(a, b | x, y) = 
    \begin{cases}
        \frac{1}{2} & \quad a \text{ XOR } B = x \text{ AND } y \\
        0 & \quad \text{otherwise} \\
    \end{cases}
\]

This set-up can alternatively be described using a matrix denoting the 
probabilities, sacrificing the brevity of the above definition with the ability
to more easily look at specific inputs or outputs.

\begin{center}
\begin{tabular}{l | c c c c}
  & 00 & 01 & 10 & 11 \\
  \hline
  00 & \(\frac{1}{2}\) & 0 & 0 & \(\frac{1}{2}\) \\
  01 & \(\frac{1}{2}\) & 0 & 0 & \(\frac{1}{2}\) \\
  10 & \(\frac{1}{2}\) & 0 & 0 & \(\frac{1}{2}\) \\
  11 & 0 & \(\frac{1}{2}\) & \(\frac{1}{2}\) & 0 \\
\end{tabular}
\end{center}
% subsection pr_box (end)
% section non_local_boxes (end)

% ==============================================================================

\section{Prism} % (fold)
\label{sec:prism}
% section prism (end)
% chapter related_work (end)

\newpage
\end{document}