\documentclass[report.tex]{subfiles}
\begin{document}

\chapter{Results and Evaluation} % (fold)
\label{cha:results_and_evaluation}

\section{Results} % (fold)
\label{sec:results}
Appendix \ref{cha:detailed_test_strategy_and_test_cases} provides a look at the
tests used to derive these results.

\subsection{Box Class} % (fold)
\label{sub:box_class_res}
While the tests on the \texttt{Box} class were basic, they covered all the cases
that they had to for this test. Due to the way the system works, instances of
trying to get probabilities of negative numbers or with incorrect array sizes
means that the system is broken elsewhere. This meant that I was perhaps looser
on the testing than might be encouraged, but testing in other areas should mean
that this potential pitfall is avoided.

\begin{table}[H]
    \centering
    \begin{tabular}{l | p{6cm} | l}
    Test & Expected & Result \\    
    \hline
    convertListTest & List is successfully converted with values intact
    & as expected \\
    testInputs \& testOutputs & no exceptions are thrown & as expected \\
    probTest & the returned value matches expected value & as expected \\
    probTestTwo & returned results match expected values & as expected \\
    probTestThree & returned results match expected values
    & as expected \\
    \end{tabular}
    \caption{Box class results.}
  \label{tab:box_result}
\end{table}
% subsection box_class_res (end)

\subsection{Syntax Checking} % (fold)
\label{sub:syntax_checking_res}
\begin{table}[H]
    \centering
    \begin{tabular}{l | l | l}
    File & Fails & Given Reason \\
    \hline
    \texttt{braces.qgrady} & Yes & Illegal character `\{'. \\
    \texttt{left\_bracket.qgrady} & Yes & `Error in line 5, column 5' \\
    \texttt{missing\_arrow.qgrady} & Yes & Illegal character `-' \\
    \texttt{missing\_output.qgrady} & Yes & `Error in line 2, column 1' \\
    \texttt{missing\_range.qgrady} & Yes & `Error in line 1, column 7' \\
    \texttt{missing\_semicolon.qgrady} & Yes & `Error in line 7, column 5' \\
    \texttt{missing\_semicolon\_2.qgrady} & Yes & `Error in line 4, column 1' \\
    \texttt{missing\_vars.qgrady} & Yes & `Error in line 4, column 10' \\
    \texttt{right\_bracket.qgrady} & Yes & `Error' \\
    \hline
    \end{tabular}
    \caption{Syntax checking results.}
    \label{tab:syntax_result}
\end{table}

All files failed as were expected. Later stages of the test handle files that
pass. In the cases where JFlex encountered the bug (\texttt{braces.qgrady}, and
\texttt{missing\_arrow}),
the illegal character was outlined, but not the location of it.

Conversely, while CUP errors would indicate where the problem was, it would not
provide any actual information about the error.
% subsection syntax_checking_res (end)

\subsection{Semantics Analysis} % (fold)
\label{sub:semantics_analysis}
The semantic analysis shows that files that do not match the expectations of the
semantic analyser will fail as intended. The system is able to ensure that
the distribution given matches what is defined by the ranges and number of
inputs and outputs in the system.

\begin{table}[H]
    \centering
    \begin{tabular}{l | p{6cm} | l}
    Test & Expected & Result \\    
    \hline
    validateVariablesTest & Expected exceptions are caught successfully 
    & as expected \\
    validateUnequalVariablesTest & Expected exception is caught 
    & as expected \\
    validateRowAmountsTest & Expected exception is caught & as expected \\
    validateHigherValuesTest & Expected exception is caught & as expected \\
    validateLowerValuesTest & Expected exception is caught & as expected \\
    validateGreaterRowLengthTest & Expected exception is caught & as expected \\
    validateSmallerRowLengthTest & Expected exception is caught & as expected \\
    validateGreaterRowSumsTest & Expected exception is caught & as expected \\
    validateSmallerRowSumTest & Expected exception is caught & as expected \\
    validateNonsignalling & No exception is thrown to denote box is fine & as expected \\
    validateSignalling & Exception caught as expected signalling & as expected \\
    allIsWellTest & No exceptions are thrown & as expected \\
    \end{tabular}
    \caption{Results}
  \label{tab:semantics_result}
\end{table}
% subsection semantics_analysis (end)

\subsection{File Generation} % (fold)
\label{sub:file_generation}
From the results of the file generation. The compiler is capable of producing
boxes with at least 4 inputs and outputs, and there is little reason to doubt
why 5 inputs and outputs would make it fail. It is interest to see how much the
average compile time of the system increases with more inputs. When comparing
\texttt{pr.qgrady} to \texttt{ternary.qgrady}, the increase range on the input
does not have the same type of impact in comparison.

\begin{table}[H]
  \centering
  \begin{tabular}{l | l | l | r}
    File & Compiles & Matches Prob & Time (ms)\\
    \hline
    pr.qgrady & Yes & Yes & 27.2 \\
    ternary.qgrady & Yes & Yes & 29.2 \\
    tripartite.qgrady & Yes & Yes & 47.4 \\
    quadpartit.qgrady & Yes & Yes 99 &
  \end{tabular}
  \caption{PRISM File Analysis Results.}
  \label{tab:prism_result}
\end{table}

Regrettably, due to a lack of time to do so, I was not able to perform any
interesting experiments with the produced modules. This had to be postponed
due to time constraints in the actual development of the system.
% subsection file_generation (end)
% section results (end)

\section{Evaluation} % (fold)
\label{sec:evaluation}
\subsection{System Evaluation} % (fold)
\label{sub:system_evaluation}

% subsection system_evaluation (end)

\subsection{Personal Evaluation} % (fold)
\label{sub:personal_evaluation}

% subsection personal_evaluation (end)
% section evaluation (end)
% chapter results_and_evaluation (end)

\newpage
\end{document}