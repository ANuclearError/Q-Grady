\documentclass[report.tex]{subfiles}
\begin{document}

\chapter{Results and Evaluation} % (fold)
\label{cha:results_and_evaluation}

\section{Results} % (fold)
\label{sec:results}
Appendix \ref{cha:detailed_test_strategy_and_test_cases} provides a look at the
tests used to derive these results, going into detail about the test cases and
their purposes.

\subsection{Box Class} % (fold)
\label{sub:box_class_res}
While the tests on the \texttt{Box} class were basic, they covered all the cases
that they had to for this test. Due to the way the system works, instances of
trying to get probabilities of negative numbers or with incorrect array sizes
means that the system is broken elsewhere. This meant that I was perhaps looser
on the testing than might be encouraged, but testing in other areas should mean
that this potential pitfall is avoided.

\begin{table}[H]
    \centering
    \begin{tabular}{l | p{6cm} | l}
    Test & Expected & Result \\    
    \hline
    convertListTest & List is successfully converted with values intact
    & as expected \\
    testInputs \& testOutputs & no exceptions are thrown & as expected \\
    probTest & the returned value matches expected value & as expected \\
    probTestTwo & returned results match expected values & as expected \\
    probTestThree & returned results match expected values
    & as expected \\
    \end{tabular}
    \caption{Box class results.}
  \label{tab:box_result}
\end{table}
% subsection box_class_res (end)

\subsection{Syntax Checking} % (fold)
\label{sub:syntax_checking_res}
\begin{table}[H]
    \centering
    \begin{tabular}{l | l | l}
    File & Fails & Given Reason \\
    \hline
    \texttt{braces.qgrady} & Yes & Illegal character `\{'. \\
    \texttt{left\_bracket.qgrady} & Yes & `Error in line 5, column 5' \\
    \texttt{missing\_arrow.qgrady} & Yes & Illegal character `-' \\
    \texttt{missing\_output.qgrady} & Yes & `Error in line 2, column 1' \\
    \texttt{missing\_range.qgrady} & Yes & `Error in line 1, column 7' \\
    \texttt{missing\_semicolon.qgrady} & Yes & `Error in line 7, column 5' \\
    \texttt{missing\_semicolon\_2.qgrady} & Yes & `Error in line 4, column 1' \\
    \texttt{missing\_vars.qgrady} & Yes & `Error in line 4, column 10' \\
    \texttt{right\_bracket.qgrady} & Yes & `Error' \\
    \hline
    \end{tabular}
    \caption{Syntax checking results.}
    \label{tab:syntax_result}
\end{table}

All files failed as were expected. Later stages of the test handle files that
pass. In the cases where JFlex encountered the bug (\texttt{braces.qgrady}, and
\texttt{missing\_arrow}),
the illegal character was outlined, but not the location of it.

Conversely, while CUP errors would indicate where the problem was, it would not
provide any actual information about the error.
% subsection syntax_checking_res (end)

\subsection{Semantics Analysis} % (fold)
\label{sub:semantics_analysis}
The semantic analysis shows that files that do not match the expectations of the
semantic analyser will fail as intended. The system is able to ensure that
the distribution given matches what is defined by the ranges and number of
inputs and outputs in the system.

\begin{table}[H]
    \centering
    \begin{tabular}{l | p{6cm} | l}
    Test & Expected & Result \\    
    \hline
    validateVariablesTest & Expected exceptions are caught successfully 
    & as expected \\
    validateUnequalVariablesTest & Expected exception is caught 
    & as expected \\
    validateRowAmountsTest & Expected exception is caught & as expected \\
    validateHigherValuesTest & Expected exception is caught & as expected \\
    validateLowerValuesTest & Expected exception is caught & as expected \\
    validateGreaterRowLengthTest & Expected exception is caught & as expected \\
    validateSmallerRowLengthTest & Expected exception is caught & as expected \\
    validateGreaterRowSumsTest & Expected exception is caught & as expected \\
    validateSmallerRowSumTest & Expected exception is caught & as expected \\
    validateNonsignalling & No exception is thrown to denote box is fine & as expected \\
    validateSignalling & Exception caught as expected signalling & as expected \\
    allIsWellTest & No exceptions are thrown & as expected \\
    \end{tabular}
    \caption{Results}
  \label{tab:semantics_result}
\end{table}
% subsection semantics_analysis (end)

\subsection{File Generation} % (fold)
\label{sub:file_generation}
From the results of the file generation. The compiler is capable of producing
boxes with at least 4 inputs and outputs, and there is little reason to doubt
why 5 inputs and outputs would make it fail. It is interest to see how much the
average compile time of the system increases with more inputs. When comparing
\texttt{pr.qgrady} to \texttt{ternary.qgrady}, the increase range on the input
does not have the same type of impact in comparison.

\begin{table}[H]
  \centering
  \begin{tabular}{l | l | l | r}
    File & Compiles & Matches Prob & Time (ms)\\
    \hline
    pr.qgrady & Yes & Yes & 27.2 \\
    ternary.qgrady & Yes & Yes & 29.2 \\
    tripartite.qgrady & Yes & Yes & 47.4 \\
    quadpartit.qgrady & Yes & Yes & 99
  \end{tabular}
  \caption{PRISM File Analysis Results.}
  \label{tab:prism_result}
\end{table}

Regrettably, due to a lack of time to do so, I was not able to perform any
interesting experiments with the produced modules. This had to be postponed
due to time constraints in the actual development of the system.
% subsection file_generation (end)
% section results (end)

\section{Evaluation} % (fold)
\label{sec:evaluation}
\subsection{System Evaluation} % (fold)
\label{sub:system_evaluation}
Based on the testing I have done with the system, the Q'Grady language and
compiler provide an acceptable platform for studying non-local boxes. While
there are additional complexities that can be explored in the extension of the
language, it is clear that the system is able to handle boxes of various types.

In terms of features, perhaps the most glaring fault is in the error handling of
the parsing stage, where the reasons of why the system failed to compile the
input file are not clear to the user. In addition, the error handling of
encountering signalling suffers similarly with a lack of information.

While perhaps being a Pandora's box in terms of future maintainability, the file
generation part of the software does handle the exponential lines of code it
must generate as inputs and outputs increase in quantity. While the runtime is
affected, the compilation is still comfortably within a reasonable time in the
covered cases.

The system is able to correctly produce simple boxes like the PR box, but is
also able to handle boxes that go beyond the binary digits or have any number
of inputs and outputs.
% subsection system_evaluation (end)

\subsection{Personal Evaluation} % (fold)
\label{sub:personal_evaluation}
The initial scope of this project was to create the Q'Grady language and its
compiler, and then use it in experiments using PRISM software. This is perhaps
the primary disappointment of the project. Due to struggles in the early stages
of the project, and insufficient communication on my part, I did not have a
working handmade PRISM model until the later stages of development. This meant
that the file generation stage was re-written on numerous occasions before I
started to consider more than two inputs and outputs, which was a dangerously
late addition to the project.

This was the first time I had worked on a singular project for an extended
period of time. It is clear that I was not perhaps as prepared coming into the
start of 2016 as would have been required for this project. Fortnightly meetings
with my supervisor proved to perhaps be not as much communication as what was
needed for the project.

In addition, while I am well acquainted with Git, GitHub's issue tracking was
used less frequently as the project went on. It is clear that for future
projects, more structure and discipline is required of me.
% subsection personal_evaluation (end)
% section evaluation (end)
% chapter results_and_evaluation (end)

\newpage
\end{document}