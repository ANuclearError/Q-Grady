\documentclass[report.tex]{subfiles}
\begin{document}

\chapter{Results and Evaluation} % (fold)
\label{cha:results_and_evaluation}

\section{Results} % (fold)
\label{sec:results}
All the tests done as part of the system were successful, although it must be
noted that not all use cases desired for the system were accomplished.

In terms of the semantic analysis. The results of Tables \ref{tab:box_summary}
and \ref{tab:semantics_summary} show that the JUnit tests are successful in
preventing forbidden boxes from being generated.

The PRISM files tested were shown to produce PRISM models that matched the
original distribution in the PRISM software. 

Simulating paths of these models also shows that in the case of there being more
inputs than outputs, a loop would be entered in the simulated paths with the
input constantly performing its synchronized action despite the box not paying
attention to it. If there was an output in the box still to be decided, this
loop would be avoided.

The following tables from Appendix
\ref{cha:detailed_test_strategy_and_test_cases} outline the results of testing
done with the system.

\begin{table}[H]
    \centering
    \begin{tabular}{l | p{6cm} | l}
    Test & Expected & Result \\    
    \hline
    convertListTest & List is successfully converted with values intact
    & as expected \\
    testInputs \& testOutputs & no exceptions are thrown & as expected \\
    probTest & the returned value matches expected value & as expected \\
    probTestTwo & returned results match expected values & as expected \\
    probTestThree & returned results match expected values
    & as expected \\
    \end{tabular}
    \caption{Results}
  \label{tab:box_result}
\end{table}

\begin{table}[H]
    \centering
    \begin{tabular}{l | p{6cm} | l}
    Test & Expected & Result \\    
    \hline
    validateVariablesTest & Expected exceptions are caught successfully 
    & as expected \\
    validateUnequalVariablesTest & Expected exception is caught 
    & as expected \\
    validateRowAmountsTest & Expected exception is caught & as expected \\
    validateHigherValuesTest & Expected exception is caught & as expected \\
    validateLowerValuesTest & Expected exception is caught & as expected \\
    validateGreaterRowLengthTest & Expected exception is caught & as expected \\
    validateSmallerRowLengthTest & Expected exception is caught & as expected \\
    validateGreaterRowSumsTest & Expected exception is caught & as expected \\
    validateSmallerRowSumTest & Expected exception is caught & as expected \\
    validateNonsignalling & No exception is thrown to denote box is fine & as expected \\
    validateSignalling & Exception caught as expected signalling & as expected \\
    allIsWellTest & No exceptions are thrown & as expected \\
    \end{tabular}
    \caption{Results}
  \label{tab:semantics_result}
\end{table}

\begin{table}[H]
  \centering
  \begin{tabular}{c | c | c | c}
    File & Compiles & Matches Prob & Time (ms)\\
    \hline
    pr.qgrady & Yes & Yes & 26.6 \\
    ternary.qgrady & Yes & Yes & 29.8 \\
    threeinputs.qgrady & Yes & Yes 30\\
  \end{tabular}
  \caption{Results}
  \label{tab:prism_result}
\end{table}

Regrettably, due to a lack of time to do so, I was not able to perform any
interesting experiments with the produced modules. This had to be postponed
due to time constraints in the actual development of the system.
% section results (end)

\section{Evaluation} % (fold)
\label{sec:evaluation}
Ultimately, the file generation of boxes beyond two inputs and outputs is
flawed. This is likely caused by not having a way to normalise the probabilities
that do not fall under `only one output is left to normalise'. In terms of
future changes to the system, this would be the priority. Attempts were made to 
rectify this situation, but it was not possible to do so. This does limit the
usefulness of the system sadly, and would take some time to fix.

I do have some ideas on how to fix this, first by changing the normalised
probability method to take an array of indices to deal with rather than just
the single index. However, the file generation would need to be severely altered
to deal with that, which is not ideal.

However, other set-ups with two inputs and outputs will still work as intended,
even if they go beyond the binary digits as their range. I do believe the
non-signalling property is correctly designed for handling larger set-ups too.
There were no situations that set-ups with more than two inputs and outputs were
unexpectedly flagged for signalling, although I do wish that I was able to
better present the source of the signalling, rather than simply showing the
user that signalling was found.
% section evaluation (end)
% chapter results_and_evaluation (end)

\newpage
\end{document}