\documentclass[report.tex]{subfiles}
\begin{document}

\chapter{Problem Description and Specification} % (fold)
\label{cha:problem_description_and_specification}

% ==============================================================================
% ==============================================================================

\section{Problem Description} % (fold)
\label{sec:problem_description}
Non-local boxes, as described in Chapter 2, can be modelled using PRISM, 
allowing for simulations and experiments to be done using the models. However,
the language is more low-level and rudimentary than would be ideal. For someone
interested in studying non-local boxes, the PRISM language may be a hindrance.

The purpose of this project is to create a language for defining the set-ups of
non-local boxes, and to implement a compiler to convert files of this new
language into PRISM models to be used for further experimentation.
% section problem_description (end)

% ==============================================================================

\section{Requirements} % (fold)
\label{sec:requirements}

\subsection{Functional} % (fold)
\label{sub:functional}
The requirements for this project demand the creation of a new language for
describing non-local boxes. It is important that the language's syntax and
semantics are organized and well meaning, allowing for easy understanding of the
language to create new set-ups.

The compiler must take in a source file of this new language, and perform
correct syntax and semantics checking on the given file. The set-up must be
checked to ensure that it is non-signalling before producing an accurate PRISM
file that defines a model of the set-up provided. It is also required that
meaningful warnings and errors are displayed to alert the user to problems
encountered during the attempted compilation.

% subsection functional (end)
\subsection{Non-Functional} % (fold)
\label{sub:non_functional}
In addition to the functional requirements, there some addition non-functional
requirements as well. Documentation should enable someone to create their own
set-ups with ease using the language, while the language should try to not rely
on PRISM knowledge as much as possible.

In addition, the compiler should run in a relatively short time, providing
documentation for the using and any future maintaining of the compiler. In
addition, it would be ideal to provide a full list of issues of an invalid
set-up rather than halting on the first issue encountered.
% subsection non_functional (end)
% section requirements (end)
% chapter problem_description_and_specification (end)
\newpage
\end{document}