\documentclass[report.tex]{subfiles}
\begin{document}

\chapter{Detailed Design and Implementation} % (fold)
\label{cha:detailed_design_and_implementation}

% ==============================================================================
% ==============================================================================

\section{Implementation Languages} % (fold)
\label{sec:implementation_languages}

\subsection{Java 8} % (fold)
\label{sub:java_8}
The compiler was created using Java version 8. The primary motivation behind
using Java was my familiarity with the language and the libraries \& frameworks
available. The complex nature of this project meant that I did not want to spend
time getting familiar with a language such as C++, while I knew that I take
advantage of a build system like Maven to handle the inevitable dependency and
testing suites I would require.
% subsection java_8 (end)

\subsection{PRISM} % (fold)
\label{sub:prism}
PRISM (\url{http://prismmodelchecker.org}) was naturally required for the
implementation of the compiler. It was necessary to ensure that the models being
generated by the compiler were valid and were accurately reflecting the set-up
specified in a .qgrady file.
% subsection prism (end)

% ==============================================================================

% subsection subsection_name (end)
% section implementation_languages (end)
\section{Tools} % (fold)
\label{sec:tools}
The Q'Grady compiler makes use of several libraries to assist in development.
These have been outlined below, outlining their function and why they were used
for the development of the compiler.

\subsection{Cup} % (fold)
\label{sub:cup}
Cup (\url{http://www2.cs.tum.edu/projects/cup}) is a parser generator that was
used to convert the Q'Grady grammar as specified in the .cup file into Java
classes that could be inserted into the Q'Grady compiler.
% subsection cup (end)

\subsection{JFlex} % (fold)
\label{sub:jflex}
JFlex (\url{http://jflex.de})is a scanner generatorfor Java. It generated an
additional Java class for the compiler, matching tokens from the scanned input
to regular expressions outlined in the .flex file.
% subsection jflex (end)

\subsection{Apache Maven} % (fold)
\label{sub:apache_maven}
Maven (\url{http://maven.apache.org}) was used primarily for handling dependency
injection. Plugins for JFlex and Cup were used to make the use of the libraries
a lot easier, since they would handle executing the generation of classes while
compiling my own Java classes. It also allowed me to automate JUnit testing
during the compilation and packaging stages.
% subsection apache_maven (end)

\subsection{Apache Commons CLI \& IO} % (fold)
\label{sub:apache_commons_cli}
The Apache Commons CLI (\url{http://commons.apache.org/proper/commons-cli}) was
used for handling the parsing of command line arguments when executing the
compiler. It was used for handling the logic behind managing the presence of
the mandatory and optional arguments for the compiler, such as the source and
destination files.

The Commons IO (\url{http://commons.apache.org/proper/commons-io}) library was
required for methods relating to handling files, such as ensuring that files
had the proper extension when required.
% subsection apache_commons_cli (end)

\subsection{JUnit} % (fold)
\label{sub:junit}
JUnit (\url{http://junit.org}) was used for testing the Java code, ensuring that
features of the compiler were accurate and doing what was intended.
% subsection junit (end)
% section tools (end)

% ==============================================================================

\section{Compiler} % (fold)
\label{sec:compiler}

\subsection{Syntax Checking} % (fold)
\label{sub:syntax_checking}
Content.
% subsection syntax_checking (end)

\subsection{Semantic Analysis} % (fold)
\label{sub:semantic_analysis}
Content.

\subsubsection{Non-Signalling} % (fold)
\label{ssub:non_signalling}
Content.
% subsubsection non_signalling (end)
% subsection semantic_analysis (end)

\subsection{Code Generation} % (fold)
\label{sub:code_generation}
Content.
% subsection code_generation (end)
% section compiler (end)
% chapter detailed_design_and_implementation (end)

% ==============================================================================
% ==============================================================================

\newpage
\end{document}