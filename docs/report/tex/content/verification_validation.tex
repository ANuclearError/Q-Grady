\documentclass[report.tex]{subfiles}
\begin{document}

\chapter{Verification and Validation} % (fold)
\label{cha:verification_and_validation}
% chapter verification_and_validation (end)
\section{Verification} % (fold)
\label{sec:verification}
In relation to the verification of the language and compiler, there are the
following considerations:
\begin{itemize}
    \item Does the compiler recognize invalid set-ups?
    \item Can the compiler produce valid PRISM models?
    \item Do the generated models match the original distribution?
\end{itemize}

\subsection{Recognizing Invalid Set-up} % (fold)
\label{sub:recognizing_invalid_set_up}
The recognizing of invalid set-ups is done with the \texttt{SemanticAnalyser}
class, so the testing of that class was of vital importance in ensuring that
the system is able to distinguish correct set-ups from incorrect set-ups. In
particular, the non-signalling property needs tested to ensure that is able to
handle the possible outcomes.
% subsection recognizing_invalid_set_up (end)

In the PRISM software, there is the feature `Compute Steady State Probabilities'
that can be used to determine whether the generated models match the Q'Grady
file. By dividing each probability in the distribution by the total number of
rows that create the distribution, we can compare that to the steady state
probabilities to determine whether or not the distribution of the input Q'Grady
file has been converted into the equivalent PRISM model.
% section verification (end)

\section{Validation} % (fold)
\label{sec:validation}

% section validation (end)
\newpage
\end{document}