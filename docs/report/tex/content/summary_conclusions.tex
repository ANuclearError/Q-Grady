\documentclass[report.tex]{subfiles}
\begin{document}

\chapter{Summary and Conclusions} % (fold)
\label{cha:summary_and_conclusions}
\section{Summary} % (fold)
\label{sec:summary}
To summarise, the development of Q'Grady saw the creation of a language that can
define the set-up of a singular non-local box. The compiler is capable of
parsing boxes of any size and range, with its semantic analysis being able to
determine signalling on these boxes as well as the addition constraints that
need to be adhered to. Tests showed that the PRISM models created by the
compiler were matching expectations as well, although I would suspect that there
are a couple of instabilities hidden at this moment in time.

Time constraints prevented the possibility of using the produced models for
further investigation, although I believe that my work will help those who wish
to do so.
% section summary (end)

\subsection{Future Considerations} % (fold)
\label{sub:future_considerations}
The following outline some improvements or additional features that could
be explored in the future.

\subsubsection{Multiple Boxes Interaction} % (fold)
\label{ssub:multiple_boxes_interaction}
Currently, the Q'Grady language only allows for creating the set-up of one box
per file. Minor changes to the CUP file could make the the parser return a list
of \texttt{Box} objects rather than just the one, with the semantics analysis
including allowing outputs of one box to be inputs of another (provided there is
no cycle).
% subsubsection multiple_boxes_interaction (end)

\subsubsection{Semantics Overhaul} % (fold)
\label{ssub:semantics_overhaul}
The formal semantics are not currently up to a standard I consider satisfactory,
due to me not giving them a high priority over the development. Future work with
the language would require a more talented person than I am to produce well
meaning denotational semantics, that I feel is lacking currently.
% subsubsection semantics_overhaul (end)

\subsubsection{Syntax Error Handling} % (fold)
\label{ssub:syntax_error_handling}
The error handling of the CUP and JFlex files could be improved up, showing
suggestions for fixing the set-up.
% subsubsection syntax_error_handling (end)

\subsubsection{Mapping Inputs to Outputs} % (fold)
\label{ssub:mapping_inputs_to_outputs}
Currently, Q'Grady assumes a 1 to 1 mapping from inputs to outputs, similar to
the hypothetical Alice and Bob experiments outlined in Section
\ref{sec:related_nlbs}. The language could be extended to also allow for
describing set-ups similar to as if Alice and Bob had two inputs for a single
output. The non-signalling check would need to be changed to facilitate this.
% subsubsection mapping_inputs_to_outputs (end)

% subsection future_considerations (end)

\section{Conclusions} % (fold)
\label{sec:conclusions}
Overall, I am moderately pleased with the destination of this project. While it
is disappointing that no experimentation using the models could be done as part
of the project, I am extremely relieved that the system is able to function with
complex set-ups now.

Since this was the first time I've dealt with working on a sole individual
project of a much larger scale than previous years, I am not surprised that the
project did suffer from less than stellar management on my part. The formal
semantics are certainly not to the same level that I had initially wanted, which
is disappointing.

Nevertheless, I am pleased that I was able to create the product that I set out
to do, even if I did not accomplish all the goals. I would consider this a
moderate or minor success overall, and the project proved to be as challenging
and interesting as I expected.
% section conclusions (end)
% chapter summary_and_conclusions (end)

\newpage
\end{document}