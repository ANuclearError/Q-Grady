\documentclass[report.tex]{subfiles}
\begin{document}

\chapter{System Design} % (fold)
\label{cha:system_design}
% chapter system_design (end)
\section{Q'Grady Language} % (fold)
\label{sec:q_grady_language}
The Q'Grady language is designed to describe the set-up of non-local boxes.
This is achieved through defining the probability distribution of that set-up.
The intent was to focus on providing a clear syntax to represent the distribution
so that the probabilities could be easily identified.
Listing \ref{pr_qgrady_example} provides the Q'Grady file for a standard PR-box.

\lstinputlisting[numbers=left, basicstyle=\ttfamily\footnotesize,
caption={The PR Q'Grady file.}, captionpos=b, label=pr_qgrady_example]
{files/pr.qgrady} 

Examples of previous prototypes are found in the Detailed Design chapter.
When the general style for the language was decided upon, the next step was to
formalise the language, providing a grammar and the semantics of the language,
in order to provide a specification that was easy to understand and work with.

\subsection{Syntax} % (fold)
\label{sub:syntax}
% subsection syntax (end)

\subsection{Semantics} % (fold)
\label{sec:semantics}

% subsection semantics (end)
% section q_grady_language (end)

\section{Q'Grady Compiler} % (fold)
\label{sec:q_grady_compiler}

% section q_grady_compiler (end)
\newpage
\end{document}