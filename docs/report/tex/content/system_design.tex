\documentclass[report.tex]{subfiles}
\begin{document}

\chapter{System Design} % (fold)
\label{cha:system_design}
% chapter system_design (end)
\section{Q'Grady Language} % (fold)
\label{sec:q_grady_language}
The Q'Grady language is designed to describe the set-up of non-local boxes. This
is achieved through defining the probability distribution of that set-up. The
intent was to focus on providing a clear syntax to represent the distribution so
that the probabilities could be easily identified. My original designs did not
achieve this aim, my attempts to divide the distribution and separate each row 
only became unclear. Examples of previous prototypes are found in the Detailed
Design chapter.

The following is a Q'Grady file representing the PR box:
\begin{center}
[0.5, 0, 0, 0.5; 0.5, 0, 0, 0.5; 0.5, 0, 0, 0.5; 0, 0.5, 0.5, 0;]
\end{center}

\subsection{Syntax} % (fold)
\label{sub:syntax}

% subsection syntax (end)

\subsection{Semantics} % (fold)
\label{sec:semantics}

% subsection semantics (end)
% section q_grady_language (end)

\section{Q'Grady Compiler} % (fold)
\label{sec:q_grady_compiler}

% section q_grady_compiler (end)
\newpage
\end{document}