\documentclass[report.tex]{subfiles}
\begin{document}
\chapter{Introduction} % (fold)
\label{cha:introduction}
Non-local boxes are theoretical objects where multiple parties share some
correlation, introduced by Sandu Popescu and Daniel Rohrlich as part of the
understanding of quantum non-locality. They can be represented as a probability
distribution with a set of inputs and outputs.

The model checker PRISM can be used to simulate non-local boxes, but manually creating these models is not the best approach.

\section{Project Objectives} % (fold)
\label{sec:project_objectives}
The goal of this project was to define a language, named Q'Grady, for describing
the set-up of non-local boxes, distributed objects with some correlation, and a
compiler to convert the language into PRISM models for experiments.

The language would be able to allow for set-ups of arbitrary size and ranges of
values, with its syntax and semantics. The compiler would take Q'Grady files and
analyse the set-up to determine its validity before converting the set-up into a
working PRISM model.

Once completed, experimentation using PRISM was to be undertaken afterwards.
% section project_objectives (end)
\section{Project Outcomes} % (fold)
\label{sec:project_outcomes}
The Q'Grady language and compiler are capable of handling different set-ups of
non-local boxes, whether they have 2 or 4 inputs or whether each input is a
binary or ternary digit. The compiler is also to identify set-ups that cannot be
correctly modelled, with mostly meaningful error messages to assist the user.

The development of the system took much longer than originally anticipated, with
a lot of time being spent on trying to make the system work on this arbitrary
set-ups. This meant that there was not enough time to perform any interesting
experiments using PRISM. In addition, the focus on the development saw the
more formalised semantics be neglected in favour of the compiler.
% section project_outcomes (end)

\section{Report Structure} % (fold)
\label{sec:report_structure}
% section report_structure (end)
The report starts with background knowledge on the theory behind non-local boxes
and PRISM, with the aim to provide the prerequisite knowledge required for the
project using some examples. This is then followed by the specification and
requirements of the project, going into further details about the goals.

The report then moves onto the design and implementation of the language and
compiler, highlighting the steps made in the creation of the system, and the
challenges faced during the development.

Testing and results then follow, with an evaluation on the system. This is then
completed with a conclusion summarising the report, as well as outlining any
possible future milestones for the system.
% chapter introduction (end)
\newpage
\end{document}