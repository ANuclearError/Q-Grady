\documentclass[report.tex]{subfiles}
\begin{document}

\chapter{Detailed Test Strategy and Test Cases} % (fold)
\label{cha:detailed_test_strategy_and_test_cases}
The Q'Grady testing saw a combination of JUnit and PRISM to determine the
quality of the produced system. JUnit was gradually added on a case by case
system, with the PRISM testing coming at the later stages of development to
specifically look at the file generation stage.

\section{JUnit Test Cases} % (fold)
\label{sec:junit_test_cases}
The JUnit Test Cases can be split into two groups, the \texttt{Box} tests and
the \texttt{SemanticAnalyser} tests. These two classes were the focus of JUnit
testing due to their calculations being the important part of the code.

\subsection{Box Class} % (fold)
\label{sub:box_class}
\subsubsection{Test Preparation} % (fold)
\label{ssub:test_preparation}
A PR box distribution is created, with a Box object created with the inputs
`x' & `y' and the outputs `a' & `b', each with ranges of 2.
% subsubsection test_preparation (end)

\subsubsection{Test List Conversion} % (fold)
\label{ssub:test_list_conversion}
\begin{itemize}
    \item Description
    \begin{itemize}
        \item Test conversion of \texttt{List<List<Double>} to
        \texttt{double[][]}
    \end{itemize}
    \item Prerequisites
    \begin{itemize}
        \item List is not null.
    \end{itemize}
    \item Expected Result
    \begin{itemize}
        \item The array matches the list.
    \end{itemize}
    \item Actual Result
    \begin{itemize}
        \item Result as expected.
    \end{itemize}
\end{itemize}
% subsubsection test_list_conversion (end)

\subsubsection{Test Get Prob} % (fold)
\label{ssub:test_get_prob}
\begin{itemize}
    \item Description
    \begin{itemize}
        \item Test \texttt{getProb()} method.
    \end{itemize}
    \item Prerequisites
    \begin{itemize}
        \item Box and prob are not null.
    \end{itemize}
    \item Expected Result
    \begin{itemize}
        \item The array matches the given array in constructor.
    \end{itemize}
    \item Actual Result
    \begin{itemize}
        \item Result as expected.
    \end{itemize}
\end{itemize}
% subsubsection test_get_prob (end)

\subsubsection{Test Input & Output} % (fold)
\label{ssub:test_input_output}
\begin{itemize}
    \item Description
    \begin{itemize}
        \item Ensures the Box's stored information is accurate.
    \end{itemize}
    \item Prerequisites
    \begin{itemize}
        \item Box and its fields are not null.
    \end{itemize}
    \item Expected Result
    \begin{itemize}
        \item The arrays of variables are equal, their sizes are equal and their
        ranges are equal to those given in the constructor.
    \end{itemize}
    \item Actual Result
    \begin{itemize}
        \item Result as expected.
    \end{itemize}
\end{itemize}
% subsubsection test_input_output (end)

\subsection{Probability Test} % (fold)
\label{sub:probability_test}
\begin{itemize}
    \item Description
    \begin{itemize}
        \item Test the accuracy of the \texttt{getProb()} method, that the right
        indices are looked at.
    \end{itemize}
    \item Prerequisites
    \begin{itemize}
        \item Box and fields are not null.
    \end{itemize}
    \item Expected Result
    \begin{itemize}
        \item The probability got from array matches expected value.
    \end{itemize}
    \item Actual Result
    \begin{itemize}
        \item Result as expected.
    \end{itemize}
\end{itemize}
% subsection probability_test (end)

\subsubsection{Testing Reduced Probability} % (fold)
\label{ssub:testing_reduced_probability}
\begin{itemize}
    \item Description
    \begin{itemize}
        \item Test the accuracy of reduced probability calculations. Two
        indices and their values are given to the reduced probability method,
        and the given result is compared to the expected result.
    \end{itemize}
    \item Prerequisites
    \begin{itemize}
        \item Box and fields are not null.
    \end{itemize}
    \item Expected Result
    \begin{itemize}
        \item The probability returned matches expected value.
    \end{itemize}
    \item Actual Result
    \begin{itemize}
        \item Result as expected.
    \end{itemize}
\end{itemize}

% subsubsection testing_reduced_probability (end)

\subsubsection{Testing Normalised Probability} % (fold)
\label{ssub:testing_normalised_probability}
\begin{itemize}
    \item Description
    \begin{itemize}
        \item Test the accuracy of normalised probability calculations. Two
        indices and their values are given to the normalised probability method,
        and the given result is compared to the expected result.
    \end{itemize}
    \item Prerequisites
    \begin{itemize}
        \item Box and fields are not null.
    \end{itemize}
    \item Expected Result
    \begin{itemize}
        \item The probability returned matches expected value.
    \end{itemize}
    \item Actual Result
    \begin{itemize}
        \item The method only works for when there is only one output still
        unknown.
    \end{itemize}
\end{itemize}
% subsubsection testing_normalised_probability (end)
% subsection box_class (end)

\subsection{SemanticAnalysis} % (fold)
\label{sub:semanticanalysis}

% subsection semanticanalysis (end)
% section junit_test_cases (end)

\section{PRISM Test Cases} % (fold)
\label{sec:prism_test_cases}

% section prism_test_cases (end)
% chapter detailed_test_strategy_and_test_cases (end)
\newpage
\end{document}