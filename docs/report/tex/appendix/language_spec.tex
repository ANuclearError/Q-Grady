\documentclass[report.tex]{subfiles}
\begin{document}

\chapter{Q'Grady Language Specification} % (fold)
\label{cha:q_grady_specification}
\section{Q'Grady Syntax} % (fold)
\label{sec:q_grady_syntax}
\lstinputlisting[numbers=left, basicstyle=\ttfamily\footnotesize,
caption={Q'Grady BNF}, captionpos=b, label=qgrady_bnf]
{files/bnf.txt} 
% section q_grady_syntax (end)

\section{Q'Grady Semantics} % (fold)
\label{sec:q_grady_semantics}
A Q'Grady box is a 5-tuple \(\{\delta, i, o, i_r, o_r\}\) where:

\(i = \{0, 1, ... n\}\)

\(o = \{0, 1, ... n\}\)

\(i_r \in \mathbb{I}\)

\(o_r \in \mathbb{I}\)

\(\delta\) = a \({i_r}^{i} x {o_r}^{o}\) matrix containing some probability 
distribution that is non-signalling.


The box is a conditional probability distribution that satisfies non-signalling.
Each row is a list of the probabilities of each outcome based on the given
input. For Listing \ref{lst:pr_qgrady_example}, line 5 would represent 
\(P(ab | 00)\). In order for the box to make sense, the following criteria must
be met:
\begin{itemize}
    \item The number of rows in the matrix must equal \({i_r}^{i}\) where
    \(range\) is the range of values an input can be, and \(size\) is the number
    of inputs.
    \item Each row's length must equal \(range ^{size}\) where \(range\) is
    the range of values an output can be, and \(size\) is the number of outputs.
    \item Each probability in the matrix be between 0 and 1.
    \item Each row of probabilities must sum to 1.
    \item There must be no duplicate variables or use of PRISM keywords in
    either the input or output variable lists.
\end{itemize}
% section q_grady_semantics (end)
% chapter q_grady_specification (end)
\newpage
\end{document}