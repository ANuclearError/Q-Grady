\documentclass[report.tex]{subfiles}
\begin{document}

\chapter{User Guide} % (fold)
\label{cha:user_guide}
This user guide will provide information on how to create Q'Grady files to
describe non-local boxes and then compile them into PRISM models. This user
guide assumes familiarity with the use of PRISM.

\section{System Requirements} % (fold)
\label{sec:system_requirements}
In order to run the Q'Grady compiler and use the produced models, the following
software is required:

\begin{itemize}
    \item Java JDK 1.8+ \footnote{\url{http://www.oracle.com/technetwork/java/javase/downloads/index.html}}
    \item PRISM 4.3 \footnote{\url{http://www.prismmodelchecker.org/}} (Note, Windows binary requires 32 bit Java)
\end{itemize}
% section system_requirements (end)

\section{Writing a Q'Grady File} % (fold)
\label{sec:writing_a_q_grady_file}
Let's start with a simple box with the following properties:
\begin{itemize}
    \item Two inputs: \(x, y \in {0, 1}\).
    \item Two inputs: \(a, b \in {0, 1}\).
    \item The PR probability distribution:
    \(
        P(a, b | x, y) = 
        \begin{cases}
            \frac{1}{2} & \quad a \text{ XOR } B = x \text{ AND } y \\
            0 & \quad \text{otherwise} \\
        \end{cases}
    \)
\end{itemize}

The finished file will look like this:

\begin{lstlisting}
    input range = 2;
    output range = 2;

    x, y -> a, b [
        0.5, 0.0, 0.0, 0.5;
        0.5, 0.0, 0.0, 0.5;
        0.5, 0.0, 0.0, 0.5;
        0.0, 0.5, 0.5, 0.0;
    ]
\end{lstlisting}

\subsection{Configuring Ranges} % (fold)
\label{sub:configuring_ranges}
Open up your preferred text editor, and let's begin. 

Every Q'Grady file starts with the same way, where the range of the inputs
and outputs are specified. There is currently no ability to specify specific
ranges per variable. Note that the given range is the number of unique values
the variables can be, not the maximum value. Think of it as a number base, since
we want binary numbers, we are using base 2.

\begin{lstlisting}
    input range = 2;
    output range = 2;
\end{lstlisting}
% subsection configuring_ranges (end)

\subsection{Configuring Variables} % (fold)
\label{sub:configuring_variables}
Again, very straight forward no matter how many inputs or outputs you have.
The variables are made up of two lists, the input and output, separated by the
`->' arrow. Each lists' members are separated by a comma.

\begin{lstlisting}
    x, y -> a, b
\end{lstlisting}

With the current Q'Grady version, please ensure that the input and output are
the same length. The system is not guaranteed to work if this is not the case.
% subsection configuring_variables (end)

\subsection{Probability Distribution} % (fold)
\label{sub:probability_distribution}
This is where the meat of the Q'Grady file is. When dealing with larger boxes,
you must take great care in ensuring the accuracy of the distribution, otherwise
you will not get the results you were looking for.

In this case, we have two binary inputs and outputs. This means our distribution
is a 4x4 matrix, totalling to 16 elements.

\begin{lstlisting}
    [
        0.5, 0.0, 0.0, 0.5;
        0.5, 0.0, 0.0, 0.5;
        0.5, 0.0, 0.0, 0.5;
        0.0, 0.5, 0.5, 0.0;
    ]    
\end{lstlisting}

Note that Q'Grady does not particularly care about white-space, meaning that so
long as the first character after the variables is the left square bracket, the
compiler will not care about any gap.

At this stage, please ensure to make any undesired outcomes of this set-up `0.0'
and \emph{not} `0'. The compiler treats these types of numbers very differently,
and only expects the former here. In addition, ensure that all rows add up to 1.
Assuming you've not accidentally created a signalling box, then the file is
ready for compilation. Save the file under any name you want with the extension
`\texttt{.qgrady}' and now we can move onto compiling.

% subsection probability_distribution (end)
% section writing_a_q_grady_file (end)
\section{Compiling the Box} % (fold)
\label{sec:compiling_the_box}
\subsection{Usage} % (fold)
\label{sub:usage}
\begin{lstlisting}
    usage: qgrady -f <file> [-o <file>] [-h] [-v]
     -f,--file <file>     takes input from <file>
     -o,--output <file>   places output to <file>
     -h,--help            prints this message
     -v,--version         displays compiler information version    
\end{lstlisting}
% subsection usage (end)

\subsection{Running the Compiler} % (fold)
\label{sub:running_the_compiler}
Assuming the source file and compiler are in the same location, you can execute
the compilation process as follows:
\begin{lstlisting}
    java -jar q.grady.jar -f filename.qgrady
\end{lstlisting}

Where `filename' is whatever you saved the file as. If you want to give the
PRISM model a specific destination, you can use the `-o' or `--output' flag as
follows:

\begin{lstlisting}
    java -jar q.grady.jar -f filename.qgrady -o dest.prism
\end{lstlisting}

Again changing the file's name as appropriate. The help and version flags, as
their name implies, will simply display the relevant information and then cease.
These flags override the compilation process.

You will see the console printing a series of checks before starting the file
writing. These are the semantics checks imposed to ensure that only valid
boxes are being fed into the compiler. If all goes well, you should see the
following output:

\begin{lstlisting}
    Checking variables... OK!
    Checking values... OK!
    Checking number of rows... OK!
    Checking row lengths... OK!
    Checking row sums... OK!
    Checking for non-signalling... OK!
    Writing box to pr.prism... OK!
\end{lstlisting}
% subsection running_the_compiler (end)

The PRISM model should now be available at it's specified location. If no
specific destination was chosen, it will simply appear in the same directory as
the source file.
% section compiling_the_box (end)

\section{Using PRISM} % (fold)
\label{sec:using_prism}
Now that we have the model, we can now use it in PRISM. At this stage, the PRISM
manual\footnote{\url{http://www.prismmodelchecker.org/manual/}} would be of use
if you are unfamiliar with PRISM. This user guide will only go over using PRISM
for ensuring the result is to your expectation.

Open the file in PRISM, and enter `Model -> Compute -> Steady State
Probabilities' in the menu bar, or press \texttt{f4} instead. The `Log' tab
should display information about the computation it has just performed. The
results should look like this:

\begin{lstlisting}
    26:(0,0,true,0,0)=0.125
    28:(0,0,true,1,1)=0.125
    39:(0,1,true,0,0)=0.125
    41:(0,1,true,1,1)=0.125
    57:(1,0,true,0,0)=0.125
    59:(1,0,true,1,1)=0.125
    70:(1,1,true,0,1)=0.125
    72:(1,1,true,1,0)=0.125
\end{lstlisting}

The first numbers indicate the state the model is in, with the variables'
values listed in the parenthesis, with the value of \texttt{ready} separating
the inputs from the outputs. As you can see from the results, the 8 states
listed in the results match the same 8 input and output pairs that had a 0.5
probability, which is quartered to 0.125 since there are 4 possible inputs in
total.

Now you are free you make more complex models and study them further using
PRISM.
% section using_prism (end)

\section{Troubleshooting} % (fold)
\label{sec:troubleshooting}
\begin{itemize}
    \item The compiler is saying `Signalling found.'
    \begin{itemize}
        \item Check your input file and ensure that there are no obvious areas
        in the matrix that suggest signalling. Unfortunately, the compiler
        cannot give you specific detail.
    \end{itemize}
    
\end{itemize}
% section troubleshooting (end)
% chapter user_guide (end)

\newpage
\end{document}