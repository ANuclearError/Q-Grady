\documentclass[11pt, a4paper]{article}
\usepackage[utf8]{inputenc}
\usepackage[parfill]{parskip}
\usepackage{url}
\setlength\parindent{0pt}

\begin{document}
\title{A Compiler for PRiSM \\
\large{Project Scope and Outline Plan}}
\author{Aidan O'Grady - 201218150\\Supervisor: Ross Duncan}
\date{}
\maketitle

\section{Overview} % (fold)
\label{sec:overview}
Non-local boxes in quantum theory are distributed objects that share some
correlations (classical or quantum). We could model them using the probabalistic
model checker PRiSM, which allows modelling and analysis of systems with random
behaviour.

However, due to the rudimentary nature of PRiSM's language, there are
limitations in defining these boxes. The aim of this prokect is to define a
language describing setups of non-local boxes. In addition, a compiler for this
new language will be developed to create PRiSM models from this language.

% section overview (end)

\section{Objectives} % (fold)
\label{sec:objectives}
For the puposes of this project, there are ofur major high-level objectives to
be tackled in order to produce a quality language and compiler:
\begin{enumerate}
    \item Studying PRiSM, its language and creating models for it.
    \item Studying quantum theory behind non-local boxes, and implementating
    in standard PRiSM.
    \item Defining my own language for non-local boxes.
    \begin{enumerate}
        \item Constructing syntax/semantics of language.
    \end{enumerate}
    \item The implementation of a compiler that will take input source files of
    my own language and produce valid PRiSM files.
    \begin{enumerate}
        \item Deciding on a suitable language for this compiler to be written
        in.
    \end{enumerate}
\end{enumerate}

% section objectives (end)

\section{Related Works} % (fold)
\label{sec:related_works}
So far, the following works have been identified:
\begin{itemize}
    \item Non-local Boxes\cite{nlb_lamontague} - A literature review about
    non-local boxes, providing definitions and also providing references to
    extra reading.
    \item Probabilistic Model Checking Lectures\cite{prism_lectures} - A series
    of lectures to help in understanding of PRiSM, to allow me to structure
    non-local boxes in PRiSM.
    \item Compilers: Principles, Techniques and Tools\cite{dragon_compiler} - A
    book about compilers, with information on systems such as lexical analysers,
    which will help structure the design of my compiler.
    \item Modern Compiler Implementation in Java\cite{java_compiler} - Should I
    decide to develop compiler in Java, this book will have information about
    object orientated design that the Aho et al book does not.
    \item Programming Language Syntax and Semantics\cite{plss} - Will provide
    information on designing the language, through its syntax and semantics.
    \item Formal Syntax and Semantics of Programming Languages\cite{fsspl} -
    Will have the same uses as the D. Watt book, will be reuqired to compare and
    contrast the two books when it comes ot differences in their apporaches.
\end{itemize}

% section related_works (end)

\section{Methodology} % (fold)
\label{sec:methodology}

% section methodology (end)

\section{Evaluation} % (fold)
\label{sec:evaluation}

% section evaluation (end)

\section{Project Plan} % (fold)
\label{sec:project_plan}
The high level objects mentioned previously currently have the follwing
milestones.

\begin{center}
    \begin{tabular}{l | p{7.5cm}}
        Preliminary Date & Task \\
        \hline
        9th of December & Understanding theory of non-local boxes \& PRiSM. \\

        31st of December & Able to create non-local boxes in PRiSM. \\

        31st of January & Defining my own language for non-local boxes. \\

        9th of March & Implementing compiler from my language to PRiSM. \\
\end{tabular}
\end{center}

% section project_plan (end)

\section{Marking Scheme} % (fold)
\label{sec:marking_scheme}
The `Experimentation-based with significant software development project'
marking scheme is what we have decided to be most suitable for this project.

The creation of my language will be heavily experimental, with the syntax and
semantics allowing much room for experimentation. In addition, I feel there is
likely to be some experimentation in the compilation from my language to PRiSM,
particularly in determing how I will structure these boxes in PRiSM's language.

The compiler is going involve significant software development. Systems like the
lexical analyser are going to involve a lot of work, in their design,
implementation and testing.

% section marking_scheme (end)

\bibliographystyle{plain}
\bibliography{project-scope-outline-plan}
\end{document}