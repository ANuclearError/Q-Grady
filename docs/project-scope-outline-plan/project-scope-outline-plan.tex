\documentclass[11pt, a4paper]{article}
\usepackage[utf8]{inputenc}
\usepackage[parfill]{parskip}
\usepackage{url}
\usepackage[margin=1in]{geometry}
\usepackage{datetime}
\usepackage{hyperref}
\setlength\parindent{0pt}

\begin{document}
\title{A Compiler for PRiSM \\
\large{Project Scope and Outline Plan}}
\author{Aidan O'Grady - 201218150\\Supervisor: Ross Duncan}
\date{\today, \currenttime}
\maketitle

\section{Overview} % (fold)
\label{sec:overview}
Non-local boxes in quantum theory are distributed objects that share some
correlations (classical or quantum). We could model them using the probabilistic
model checker PRiSM, which allows modelling and analysis of systems with random
behaviour.

However, due to the rudimentary nature of PRiSM's language, there are
challenges in defining these boxes. The aim of this project is to define a
language, Q'GRADY, describing set-ups of NLBs. In addition, a compiler for this
new language will be developed to create PRiSM models from Q'GRADY files.

% section overview (end)

\section{Objectives} % (fold)
\label{sec:objectives}
For the purposes of this project, the following objectives have been identified:
\begin{enumerate}
    \item Implementing standard examples of NLBs in PRiSM, with understanding of
    debugging models within PRiSM.
    \item Understanding differences between operational and denotational
    semantics in language definitions.
    \item Designing Q'GRADY:
    \begin{enumerate}
        \item Identifying Q'GRADY's required and desired features.
        \item Defining a user friendly and concise syntax for Q'GRADY.
        \item Defining denotational semantics of Q'GRADY.
    \end{enumerate}
    \item Design and implementation of compiler:
    \begin{enumerate}
        \item Identifying the most suitable technology for implementing
        Q'GRADY's compiler.
        \item Design and implement the compiler.
        \item Test the compiler against the standard examples.
    \end{enumerate}
    \item Evaluating the equivalence of the generated PRiSM code to Q'GRADY.
    semantics.
    \item Conduct numerical experiments using the generated models (Optional).
\end{enumerate}

% section objectives (end)

\section{Related Works} % (fold)
\label{sec:related_works}
So far, the following works have been identified:
\begin{itemize}
    \item Non-local Boxes \cite{nlb_lamontague} - A literature review about
    NLBs, general overviews and definitions in addition to providing references
    to extra reading.
    \item Closed sets of non-local correlations
    \cite{Jonathan-Allcock:2009pd} - An extra article to help understanding
    of NLBs.
    \item Probabilistic Model Checking Lectures \cite{prism_lectures} - A
    series of lectures to help in understanding of PRiSM, to allow me to
    structure NLBs in PRiSM.
    \item Calculus of communicating systems \cite{ccs} - This and the related
    Wikipedia article will provide more information on PRiSM language, with
    directions to further reading as well. 
    \item Compilers: Principles, Techniques and Tools
    \cite{dragon_compiler} - A book about compilers, with information on
    systems such as lexical analysers, which will help structure the design of
    my compiler.
    \item Modern Compiler Implementation in Java \cite{java_compiler} -
    Should I decide to develop compiler in Java, this book will have information
    about object oriented compiler design that will be useful.
    \item Programming Language Syntax and Semantics \cite{plss} - Will
    provide information and guidance on designing the Q'GRADY.
    \item Formal Syntax and Semantics of Programming Languages \cite{fsspl}
    - Will have the same uses as the D. Watt book, will be required to compare
    and contrast the two books when it comes to differences in their approaches.
\end{itemize}

% section related_works (end)

\section{Methodology} % (fold)
\label{sec:methodology}
In order to have a full specification for Q'GRADY and the compiler, I must 
understand the theory behind NLBs and be able to model them myself in PRiSM.
This will be achieved through the related reading: the Lamontague and Allcock et
al papers, in addition to PRiSM lectures and CCS article. This will be
supplemented with additional learning during meetings with supervisor.

The methodology behind the language design involves studying the denotational
semantics of existing languages to allow me to structure my own for Q'GRADY,
discovering the best practises and what to avoid in the design. This will then
be moved onto the design of the compiler, whose design will be influenced by the
language itself. The sue of UML and similar diagrams will be considered as part
of the design.

The compiler will be approached with an incremental and iterative development
approach. Development will progress through various features, aimed towards
certain milestones, with the aim being to build up the system gradually until it
is complete.

The verification of the system will be testing involving using the generated
files in PRiSM, ensuring that the output is both syntactically correct, and also
accurately represents NLBs in PRiSM as well.

% section methodology (end)

\section{Evaluation} % (fold)
\label{sec:evaluation}
Evaluation will largely be based on the quality of the models that can be
produced by Q'GRADY and its compiler. For example, being able to create models
with more than the two `Alice and Bob' agents as shown in 
\cite[Definition~1]{nlb_lamontague}. In addition, the creation of models where
each has more than one input bit should also be considered for this project.

In addition, the features that are implemented will also be a factor in
evaluation. Being able to model protocols such as the Non-locality distribution
protocol \cite[Definition~16]{nlb_lamontague} is one such example.

Features such as ensuring that the model is non-signalling and the user
friendliness of Q'GRADY and its compiler will be the signs of a quality
implementation.

% section evaluation (end)

\section{Project Plan} % (fold)
\label{sec:project_plan}
The high level objects mentioned previously currently have the following
milestones. It must be noted that the report will be continuously written in
addition to these other deadlines. In the event of being ahead of schedule, the
gained time will be used to attempt the optional goal noted in section
\ref{sec:objectives}. A Gantt chart has been produced which goes into more
detail than the table below, adding in more than just a high level sections and
deadlines. It is available here:

\url{https://devweb2015.cis.strath.ac.uk/~wlb12153/project/outline-plan.png}

\begin{center}
    \begin{tabular}{l | p{7.5cm}}
        Preliminary Deadline & Task \\
        \hline
        9th of December & Understanding theory of non-local boxes \& able to
        implement them in PRiSM. \\

        & Project Specification/Plan and Project Poster due. \\

        9th of January & Full definition Q'GRADY, including syntax and
        denotational semantics. \\

        1st of February & Designing the compiler. \\

        5th of February & Progress Report 1 Due. \\

        4th of March & Progress Report 2 Due. \\

        20th of March & Implementing and verifying compiler. \\
\end{tabular}
\end{center}
% section project_plan (end)

\section{Marking Scheme} % (fold)
\label{sec:marking_scheme}
The `Experimentation-based with significant software development project'
marking scheme is what we have decided to be most suitable for this project.

The creation of my language will be heavily experimental, with the syntax and
semantics allowing much room for experimentation. In addition, I feel there is
likely to be some experimentation in the compilation from my language to PRiSM,
particularly in deciding how I will structure these boxes in PRiSM's language.

The compiler is going involve significant software development. Systems like the
lexical analyser are going to involve a lot of work, in their design,
implementation and testing. 

% section marking_scheme (end)
\newpage
\bibliographystyle{plain}
\bibliography{project-scope-outline-plan}
\end{document}