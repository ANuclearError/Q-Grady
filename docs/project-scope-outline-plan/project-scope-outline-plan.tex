\documentclass[11pt, a4paper]{article}
\usepackage[utf8]{inputenc}
\usepackage[parfill]{parskip}
\usepackage{url}
\usepackage[margin=1in]{geometry}
\setlength\parindent{0pt}

\begin{document}
\title{A Compiler for PRiSM \\
\large{Project Scope and Outline Plan}}
\author{Aidan O'Grady - 201218150\\Supervisor: Ross Duncan}
\date{}
\maketitle
\newpage

\section{Overview} % (fold)
\label{sec:overview}
Non-local boxes in quantum theory are distributed objects that share some
correlations (classical or quantum). We could model them using the probabilistic
model checker PRiSM, which allows modelling and analysis of systems with random
behaviour.

However, due to the rudimentary nature of PRiSM's language, there are
limitations in defining these boxes. The aim of this project is to define a
language, \$NAME describing set-ups of NLBs. In addition, a compiler for this new language will be developed to create PRiSM models from this language.

% section overview (end)

\section{Objectives} % (fold)
\label{sec:objectives}
For the purposes of this project, the following objectives have been identified:
\begin{enumerate}
    \item Implementing standard examples of NLBs in PRiSM, with understanding on
    debugging models within PRiSM.
    \item Understanding differences between operational and denotational
    semantic.
    \item Designing \$NAME:
    \begin{enumerate}
        \item Identifying \$NAME's required and desired features.
        \item Defining a user friendly and concise syntax for \$NAME.
        \item Defining denotational semantics of \$NAME.
    \end{enumerate}
    \item Design and implementation of compiler:
    \begin{enumerate}
        \item Identifying the most suitable technology for implementing \$NAME's
        compiler.
        \item Design and implement the compiler.
        \item Test the compiler against the standard examples.
    \end{enumerate}
    \item Evaluating the equivalence of the generated PRiSM code to \$NAME's
    semantics.
    \item Conduct numerical experiments using the generated models (Optional).
\end{enumerate}

% section objectives (end)

\section{Related Works} % (fold)
\label{sec:related_works}
So far, the following works have been identified:
\begin{itemize}
    \item Non-local Boxes\cite{nlb_lamontague} - A literature review about
    NLBs, general overviews and definitions in addition to providing references
    to extra reading.
    \item Closed sets of non-local correlations\cite{Jonathan-Allcock:2009pd} -
    An extra article to help understanding of NLBs.
    \item Probabilistic Model Checking Lectures\cite{prism_lectures} - A series
    of lectures to help in understanding of PRiSM, to allow me to structure
    non-local boxes in PRiSM.
    \item Calculus of communicating systems - This and related Wikipedia article
    will provide more information on PRiSM language, with directions to further
    reading as well. 
    \item Compilers: Principles, Techniques and Tools\cite{dragon_compiler} - A
    book about compilers, with information on systems such as lexical analysers,
    which will help structure the design of my compiler.
    \item Modern Compiler Implementation in Java\cite{java_compiler} - Should I
    decide to develop compiler in Java, this book will have information about
    object orientated design that the Aho et al book does not.
    \item Programming Language Syntax and Semantics\cite{plss} - Will provide
    information on designing the language, through its syntax and semantics.
    \item Formal Syntax and Semantics of Programming Languages\cite{fsspl} -
    Will have the same uses as the D. Watt book, will be required to compare and
    contrast the two books when it comes to differences in their approaches.
\end{itemize}

% section related_works (end)

\section{Methodology} % (fold)
\label{sec:methodology}

% section methodology (end)

\section{Evaluation} % (fold)
\label{sec:evaluation}

% section evaluation (end)

\section{Project Plan} % (fold)
\label{sec:project_plan}
The high level objects mentioned previously currently have the following.
milestones.

\begin{center}
    \begin{tabular}{l | p{7.5cm}}
        Preliminary Date & Task \\
        \hline
        9th of December & Understanding theory of non-local boxes \& able to
        implement them in PRiSM. \\

        9th of January & Defining \$NAME. \\

        9th of March & Designing & Implementing compiler from my language to
        PRiSM. \\
\end{tabular}
\end{center}

% section project_plan (end)

\section{Marking Scheme} % (fold)
\label{sec:marking_scheme}
The `Experimentation-based with significant software development project'
marking scheme is what we have decided to be most suitable for this project.

The creation of my language will be heavily experimental, with the syntax and
semantics allowing much room for experimentation. In addition, I feel there is
likely to be some experimentation in the compilation from my language to PRiSM,
particularly in deciding how I will structure these boxes in PRiSM's language.

The compiler is going involve significant software development. Systems like the
lexical analyser are going to involve a lot of work, in their design,
implementation and testing.

% section marking_scheme (end)
\newpage
\bibliographystyle{plain}
\bibliography{project-scope-outline-plan}
\end{document}