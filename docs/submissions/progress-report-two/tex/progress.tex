\documentclass[report.tex]{subfiles}
\begin{document}

\section{Progress} % (fold)
\label{cha:progress}

\subsection{The Q'Grady Compiler} % (fold)
\label{sub:the_q_grady_compiler}
\subsubsection{Non-signalling check} % (fold)
\label{sub:non_signalling_check}
A naive non-signalling check has been implemented as part of the compiler.
Currently, it is only applicable to a (2, 2, 2) set-up, meaning that it can only
handle a device two two ends, with two inputs and two outputs in total. A more
flexible alternative is being planned to try and make up for this problem.

The non-signalling check performs the below equation for both inputs. 

\[\sum_{b} P(a, b | x, y) = \sum_{b} P(a, b | x, y') = P(a | x) 
\quad \forall a, x, y, y'\]

I am currently unsure of how to best do this in a clean way, the current method
using multiple nested loops to achieve the above is naturally not the most
readable solution, although this has already been discussed with my supervisor.
My more immediate concern is getting the .prism file generated correctly before
I enhance this feature of the compiler.
% subsubsection non_signalling_check (end)

\subsubsection{Code Generation} % (fold)
\label{ssub:code_generation}
The code generation part of the compiler has not went as well as I had hoped.
The .prism file I was working towards as part of this stage was not the model
that my supervisor had wanted to be produced, meaning that any progress I
thought was being made turned out to be misguided.

Code generation is being achieved with the FileGenerator class, which contains
various strings acting as templates that can be used to plug in values from
the Box class into the file. There is the caveat that the names of variables
within the .prism file are being created by this class currently, rather than
in the .qgrady file itself, although given the time constraints currently, it is
likely that this will not be changing in the near future.

While my supervisor has tried to give me direction on the proper way to
structure the .prism file, it has been extremely difficult for me to really
grasp, which is an immediate problem that must be addressed.
% subsubsection code_generation (end)
% subsection the_q_grady_compiler (end)

\subsection{The Q'Grady Report} % (fold)
\label{sub:the_q_grady_report}
Due to the issues surrounding the development of the compiler, the report has
suffered setbacks as a result. I have started writing on the related work
section, as well as on the problem description/specification and system design.
Excerpts of these sections have been included for reading.
% subsection the_q_grady_report (end)

\subsection{Future Plans} % (fold)
\label{sub:future_plans}
At this point in time, I had wanted the compiler to be completed to enable me to
run experiments using the .prism files it would generate. Clearly, this has not
happened, and as a result, the project has fallen far behind. At this point in
time, I no longer have the confidence in myself or my abilities to complete the
project to the scope that was originally outlined. Should I complete the
compiler by the 13th of March, I would just have 2 weeks to do so, which would
not allow for the depth that would have originally been desired.
% subsection future_plans (end)

% section progress (end)
\newpage
\end{document}