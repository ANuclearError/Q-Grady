\documentclass[12pt, a4paper]{article}
\usepackage[utf8]{inputenc}
\usepackage[parfill]{parskip}
\setlength\parindent{0pt}

\begin{document}
\title{A Compiler for PRiSM \\
\large{Project Scope and Outline Plan}}
\author{Aidan O'Grady - 201218150\\{Supervisor: Ross Duncan}}
\date{}
\maketitle
\newpage

\section{Overview} % (fold)
\label{sec:overview}
Non-local boxes in quantum theory is a concept where distributed objects share
some correlations (classical or quantum). The purpose of this project is to
define a language describing setups of non-local boxes and then implement a
compiler from this new language to PRiSM.

To model these non-local boxes, the probabalistic model checker PRiSM can be
used to do so. PRiSM allows for modelling systems that exhibit random behaviour,
a trait that is found in non-local boxes.

However, due to the rudimentary nature of PRiSM's language, there are
limitations in defining these boxes. The language to be created will act as an
intermediary between the user and PRiSM, hiding these limitations from the user
creating non-local boxes.

% section overview (end)

\section{Objectives} % (fold)
\label{sec:objectives}
For the puposes of this project, there are ofur major high-level objectives to
be tackled in order to produce a quality language and compiler:
\begin{enumerate}
    \item Obtaining knowledge of PRiSM and its language.
    \item Understanding the quantum theory behind non-local boxes, and their
    implementation in standard PRiSM.
    \item Defining my own language for non-local boxes.
    \begin{enumerate}
        \item Constructing the syntax of this langiage.
        \item Constructing the semantics behind this syntax.
    \end{enumerate}
    \item The implementation of a compiler that will take input source files of
    my own language and produce valid PRiSM files.
    \begin{enumerate}
        \item Deciding on a suitable language for this compiler to be written
        in.
    \end{enumerate}
\end{enumerate}

% section objectives (end)

\section{Related Works} % (fold)
\label{sec:related_works}

% section related_works (end)

\section{Methodology} % (fold)
\label{sec:methodology}

% section methodology (end)

\section{Evaluation} % (fold)
\label{sec:evaluation}

% section evaluation (end)

\section{Project Plan} % (fold)
\label{sec:project_plan}
The high level objects mentioned previously currently have the follwing
milestones.

\begin{center}
    \begin{tabular}{l | p{7.5cm}}
        Preliminary Date & Task \\
        \hline
        9th of December & Understanding theory of non-local boxes \& PRiSM. \\

        31st of December & Able to create non-local boxes in PRiSM. \\

        31st of January & Defining my own language for non-local boxes. \\

        9th of March & Implementing compiler from my language to PRiSM. \\
\end{tabular}
\end{center}

% section project_plan (end)

\section{Marking Scheme} % (fold)
\label{sec:marking_scheme}
The `Experimentation-based with significant software development project'
marking scheme is what we have decided to be most suitable for this project.

The creation of my language will be heavily experimental, with the syntax and
semantics allowing much room for experimentation. In addition, I feel there is
likely to be some experimentation in the compilation from my language to PRiSM,
particularly in determing how I will structure these boxes in PRiSM's language.

The compiler is going involve significant software development. Systems like the
lexical analyser are going to involve a lot of work, in their design,
implementation and testing.

% section marking_scheme (end)

\end{document}