\documentclass[11pt, a4paper]{article}
\usepackage[parfill]{parskip}
\setlength\parindent{0pt}

\begin{document}
\title{Q'Grady \\A Compiler for PRiSM \\
\large{Project Specification and Plan}}
\author{Aidan O'Grady - 201218150\\Supervisor: Ross Duncan}
\date{}
\maketitle

\section{Introduction} % (fold)
\label{sec:introduction}
Non-local boxes, a concept in quantum theory, are distributed objects that share
some correlations (classical or quantum). Consider a device with two ends, one
held by Alice, the other by Bob. Given input from both Alice and Bob, they will
both receive some output based on some probability distribution \(P(a,b|x,y\)
where \(x,y,a,b \in {0,1}\).

The model checker PRISM can be used to represent these boxes in its language,
to run simulations and experiments with. However, it is a low level language and
rudimentary in its nature.

The purpose of this project is to define a domain-specific language, Q'Grady, to
describe the set-up of these non-local boxes and to implement a compiler that
will take Q'Grady files and produce valid PRISM models that can be used for
further experimentation.
% section introduction (end)


\section{Requirements} % (fold)
\label{sec:requirements}
\subsection{Functional Requirements} % (fold)
\label{sub:functional_requirements}
The functional requirements of the Q'Grady language is to provide a cohesive
syntax and semantics to describe the set-up of non-local boxes.

The functional requirements of the Q'Grady compiler is as follows:
\begin{itemize}
    \item Take Q'Grady source files as input.
    \item Parse and check source for correctness in relation to syntax and
    semantics.
    \item Check probability distribution to ensure non-signalling property
    holds.
    \item Produce equivalent PRISM file that can be used in the software.
    \item Produce meaningful warnings/errors encountered during the
    compilation.
\end{itemize}
% subsection functional_requirements (end)

\subsection{Non-Functional Requirements} % (fold)
\label{sub:non_functional_requirements}
Non-functional requirements for the compiler would be stronger integration with
PRISM. For example, it would be nice to have successful compilations be launched
in PRISM for immediate use, although this would be a addition.

Another non-functional requirement would be that the compilation takes as little
time as possible to complete. In addition, the system shall attempt to display
all errors with the Q'Grady source file, rather than just halting at the first
it encounters.
% subsection non_functional_requirements (end)
% section requirements (end)


\section{Development} % (fold)
\label{sec:development_process}

% section development_process (end)


\section{Evaluation} % (fold)
\label{sec:evaluation}

% section evaluation (end)


\section{Risks} % (fold)
\label{sec:risks}

% section risks (end)


\section{Revised Plan} % (fold)
\label{sec:revised_plan}

% section revised_plan (end)
\end{document}