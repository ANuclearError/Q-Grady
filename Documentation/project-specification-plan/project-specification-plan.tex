\documentclass[11pt, a4paper]{article}
\usepackage[parfill]{parskip}
\usepackage[margin=1in]{geometry}
\usepackage{hyperref}
\setlength\parindent{0pt}
\begin{document}
\title{Q'Grady \\A Compiler for PRiSM \\
\large{Project Specification and Plan}}
\author{Aidan O'Grady - 201218150\\Supervisor: Ross Duncan\\Second Marker: Isla
Ross}
\date{}
\maketitle

\section{Introduction} % (fold)
\label{sec:introduction}
Non-local boxes, a concept in quantum theory, are distributed objects that share
some correlations (classical or quantum). Consider a device with two ends, one
held by Alice, the other by Bob. Given input from both Alice and Bob, they will
both receive some output based on some probability distribution \(P(a,b|x,y)\)
where \(x,y,a,b \in {0,1}\).

The model checker PRISM can be used to represent these boxes in its language,
to run simulations and experiments with. However, it is a low level language and
rudimentary in its nature.

The purpose of this project is to define a domain-specific language, Q'Grady, to
describe the set-up of these non-local boxes and to implement a compiler that
will take Q'Grady files and produce valid PRISM models that can be used for
further experimentation.
% section introduction (end)


\section{Requirements} % (fold)
\label{sec:requirements}
\subsection{Functional Requirements} % (fold)
\label{sub:functional_requirements}
The functional requirements of the Q'Grady language is to provide a cohesive
syntax and semantics to describe the set-up of non-local boxes.

The functional requirements of the Q'Grady compiler is as follows:
\begin{itemize}
    \item Take Q'Grady source files as input.
    \item Parse and check source for correctness in relation to syntax and
    semantics.
    \item Check probability distribution to ensure non-signalling property
    holds.
    \item Produce equivalent PRISM file that can be used in the software.
    \item Produce meaningful warnings/errors encountered during the
    compilation.
\end{itemize}
% subsection functional_requirements (end)

\subsection{Non-Functional Requirements} % (fold)
\label{sub:non_functional_requirements}
Non-functional requirements for the Q'Grady language includes:
\begin{itemize}
    \item Providing quality documentation to demonstrate how to create set-ups
    using the language.
    \item Be as high level as possible, so that you do not need to
    have knowledge of PRISM's language to create a Q'Grady file.
\end{itemize}
Non-functional requirements for the compiler would be as follows:
\begin{itemize}
    \item A compiler that completes in an little time as possible.
    \item Provision of quality documentation on using and/or maintaining the
    compiler.
    \item Provides complete error information about a Q'Grady file, rather than
    just halting at the first found error.
\end{itemize}
% subsection non_functional_requirements (end)
% section requirements (end)


\section{Development} % (fold)
\label{sec:development}
\subsection{Development Process} % (fold)
\label{sub:development_process}
In the `Project Scope and Outline Plan' submission, I had stated that I would
be taking an `incremental and iterative development', noting that a methodology
such as Agile did not seem appropriate at the time. Since then, I have attended
seminars hosted by Amazon and Baillie Gifford, and I do now think that there
are advantages to Agile I can take advantage of when developing the compiler.

The testing of the compiler should be done concurrently with the development of
the compiler, which means that I should in theory have more time for
experimentation if I am not trying to spend time at the end testing.

Weekly sprints followed by supervisor meetings can help ensure that I am meeting
the specification of the project, so I am going astray with each new build, it
will be caught quicker.

For the development of Q'Grady, the following technologies have been identified
that will assist in the creation of the language and compiler.
% subsection development_process (end)

\subsection{PRISM} % (fold)
\label{sub:prism}
PRISM is a probabilistic model checker, software used to create formal models
using the PRISM language and allows for analysis of the models created. The
models are created to represent systems with random or probabilistic behaviour.

PRISM has been an essential part of the background study and theory learning
aspect of this project, the first of the three high-level milestones. Work has
been done in getting comfortable with the PRISM software, being able to create
models, run simulations and experiments using the software. I am now at the
stage where I can create set-ups of non-local boxes in PRISM, after some
difficulty in sufficiently grasping the concepts behind both the theory and
PRISM itself.

Since the purpose of the Q'Grady language and compiler is to create PRISM
models, the use of PRISM is a necessity as part of the testing and validation of
the compiler. There is a need for a work-flow to allow for making this easier,
although I have not yet discovered it.
% subsection prism (end)
\subsection{Languages \& Libraries} % (fold)
\label{sub:languages_libraries}
\subsubsection{Java 8} % (fold)
\label{ssub:java_8}
The compiler for the Q'Grady language will be written in Java. Java was chosen
primarily due to it being the language I am comfortable with, meaning that less
time will be spent looking for solutions to more trivial problems such as
dependency management.

While a language such as C/C++ may seem more appropriate for a compiler, since
the requirements of the compiler are to create PRISM models rather than
executable programs, I felt that there was no need to deal with aspects such as
memory management, while Java would allow me to design the compiler system
without worrying about getting used to a new environment.

In addition, the use of Java allows me to use libraries and tools that I am
already comfortable with, making the development easier.
% subsubsection java_8 (end)
% subsection languages_libraries (end)
\subsubsection{JUnit} % (fold)
\label{ssub:junit}
JUnit will be used as part of the testing strategy of the compiler. It will help
test features of the compiler such as the syntax checker, or the algorithm used
to ensure that the user's created non-local box is non-signalling.
% subsubsection junit (end)
\subsubsection{Apache Commons CLI} % (fold)
\label{ssub:apache_commons_cli}
The Apache Commons CLI library allows for easier handling of command arguments
for the compiler. As a compiler, there is naturally a requirement for an input
Q'Grady file as well as an optional destination for the PRISM file created, in
addition to any other flags that are thought of to be helpful. 

The Commons CLI library can handle all the logic behind the parsing of the
user's arguments, ensuring that the required arguments are present, and handling
mutually exclusive flags as well.
% subsubsection apache_commons_cli (end)
\subsubsection{CUP} % (fold)
CUP is a parser generator written in Java. It will use the BNF grammar that I
will create for Q'Grady, and will use it for handling the parsing of the Q'Grady
files being compiled. Due to it being closely associated with the lexer
generator JFlex (described briefly below), it is likely that I will lean towards
this parser generator unless I find extreme difficulties with its use.
\label{ssub:cup}

% subsubsection cup (end)
\subsubsection{ANTLR} % (fold)
ANTLR is an alternative to CUP. I will need to compare and contrast the two
while creating the language to determine which is more suitable for this
project.
\label{ssub:antlr}

% subsubsection antlr (end)
\subsubsection{JFlex} % (fold)
\label{ssub:jflex}
JFlex is a lexical analyser generator that will come in useful for the creation
of the compiler. It is designed to work with CUP, while also working with ANTLR
as well, meaning that I have the ability to see for myself which pairing I am
more comfortable with using. It will create a lexer based on the language
specifications I will define to be used in compilation.
% subsubsection jflex (end)

\subsection{IDEs \& Tools} % (fold)
\label{sub:ides_tools}
\subsubsection{IntelliJ IDEA} % (fold)
IntelliJ IDEA is the Java IDE I plan on using during the development of the
Q'Grady compiler. It is my preferred IDE due to being more powerful and user
friendly in my opinion.
\label{ssub:intellij_idea}
% subsubsection intellij_idea (end)
\subsubsection{Apache Maven} % (fold)
\label{ssub:apache_maven}
Maven will assist in the building process of the compiler, allowing for 
automated JUnit testing and dependency management. This will help increase the
efficiency of the development process, in addition to handling any transitive
dependencies that are required as part of the development.
% subsubsection apache_maven (end)
\subsubsection{Git} % (fold)
\label{ssub:git}
Git will be used for handling version control during the development process. I
am very comfortable with Git, while having never used systems such as SVN, so
there will be no issues with using it during the development process.
% subsubsection git (end)
% subsection ides_tools (end)
% section development (end)


\section{Evaluation} % (fold)
\label{sec:evaluation}
One part of evaluation is analysing the models created by Q'Grady. Running
experiments and simulations using the models will assist in determining how
useful Q'Grady is. The complexity of models created is another factor, the more
complex boxes can be set-up would result in more interesting experiments and
results to be obtained. The user friendliness of the language and compiler
should also be considered.
% section evaluation (end)


\section{Risks} % (fold)
\label{sec:risks}
The major risk is related to the understanding of the quantum theory. If my 
comprehension of the non-local boxes' set-ups or the criteria of non-signalling
are not correct, then the resulting language and compiler's usefulness is
thoroughly depleted.

Since this is the first time I have created my own language, then there will be
risks related to the cleanliness or readability of the language that is defined.
It is unlikely that it will be perfect right off the bat, but the better the
understanding of the theory, the easier the language will be to create.

When it comes to the development of the compiler, the primary risk will be
ensuring that I allow for sufficient time for testing and experimentation using
the PRISM models created by the compiler. THis is in addition to the continuous
report writing that must be done in addition to the development (and other
stages).
% section risks (end)


\section{Plan} % (fold)
\label{sec:plan}
A revised plan, based on the plan supplied for the `Project Scope and Outline
Plan' submission, is available here:

\url{https://devweb2015.cis.strath.ac.uk/~wlb12153/project/revised-plan.png}

A summary of the high level sections and other submissions is shown below.

It has been admittedly difficult finding the right balance for each section. I
believe that these aims are within reason, assuming there are no major
complications arising in the future.
\begin{center}
    \begin{tabular}{l | p{7.5cm}}
        Task & Estimated Completion\\
        \hline
        Creating Q'Grady Language & 3rd of January \\

        Designing the compiler & 17th of January \\

        Progress Report 1 Due & 5th of February \\

        Progress Report 2 Due & 4th of March \\

        Develop compiler & 5th of March \\

        Experimentation & 25th of March \\

        Final submission & 30th of March \\

        Project Demo & 20th of April
\end{tabular}
\end{center}
% section plan (end)
\end{document}