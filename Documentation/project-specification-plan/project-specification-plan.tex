\documentclass[11pt, a4paper]{article}
\usepackage[parfill]{parskip}
\usepackage[margin=1in]{geometry}
\setlength\parindent{0pt}

\begin{document}
\title{Q'Grady \\A Compiler for PRiSM \\
\large{Project Specification and Plan}}
\author{Aidan O'Grady - 201218150\\Supervisor: Ross Duncan}
\date{}
\maketitle

\section{Requirements} % (fold)
\label{sec:requirements}

\subsection{Functional Requirements} % (fold)
\label{sub:functional_requirements}

% subsection functional_requirements (end)

\subsection{Non-functional Requirements} % (fold)
\label{sub:non_functional_requirements}

% subsection non_functional_requirements (end)

% section requirements (end)
\newpage
\section{Technologies to be used} % (fold)
\label{sec:technologies_to_be_used}

The following identifies tools that will be used as part of the project. 
Please understand that due to the structure of this project, this is not an
exhaustive list of libraries or tools, due to the amount of time required for
background study and language design.

% subsection technologies_to_be_used (end)

\subsection{PRISM}
\label{sub:prism}
PRISM is a probabilistic model checker, software used to create formal models
using the PRISM language and allows for analysis of the models created. The
models are created to represent systems with random or probabilistic behaviour.

Since the purpose of the Q'Grady language and compiler is to create PRISM
models, the use of PRISM is a necessity as part of the testing and validation of
the compiler, in addition to the background study and as part of experimentation
using the models generated by the Q'Grady compiler.

% Need to expand this to ~half a page ideally.

% subsection prism (end)

\subsection{Java 8}
\label{sub:java_eight}
The compiler for the Q'Grady language will be written in Java. Java was chosen
primarily due to it being the language I am comfortable with, meaning that less
time will be spent looking for solutions to more trivial problems such as
dependency management.

While a language such as C/C++ may seem more appropriate for a compiler, since
the requirements of the compiler are to create PRISM models rather than
executable programs, I felt that there was no need to deal with aspects such as
memory management, while Java would allow me to design the compiler system
without worrying about getting used to a new environment.

In addition, the use of Java allows me to use libraries and tools that I am
already comfortable with.

% subsection java_eight (end)

\subsubsection{JUnit}
\label{sub:junit}
JUnit will be used as part of the testing strategy of the compiler. It will help
test features of the compiler such as the syntax checker, as well as the check
that ensures that the user's created non-local box is non-signalling.

% subsection junit (end)

\subsubsection{Apache Maven}
\label{sub:maven}
Maven will assist in the building process of the compiler, allowing for 
automated JUnit testing and dependency management. This will help increase the
efficiency of the development process, in addition to handling any transitive
dependencies that are required as part of the development.

% subsection maven (end)

\subsubsection{Apache Commons CLI}
\label{sub:commons_cli}
The Apache Commons CLI library allows for easier handling of command arguments
for the compiler. As a compiler, there is naturally a requirement for an input
Q'Grady file as well as an optional destination for the PRISM file created, in
addition to any other flags that are thought of to be helpful. 

The Commons CLI library can handle all the logic behind the parsing of the
user's arguments, ensuring that the required arguments are present, and handling
mutually exclusive flags as well.


% subsection commons_cli (end)

% section technologies_to_be_used (end)

\section{Development Process} % (fold)
\label{sec:development_process}

% section development_process (end)

\section{Evaluation} % (fold)
\label{sec:evaluation}

% section evaluation (end)

\section{Risks} % (fold)
\label{sec:risks}

% section risks (end)
\end{document}